\documentclass[11pt]{article}
\usepackage{amsmath,amsthm,amssymb}
\usepackage[margin=1in]{geometry}
\newcommand{\A}{\texttt{A}}
\newcommand{\G}{\texttt{G}}
\newcommand{\U}{\texttt{U}}
\newcommand{\C}{\texttt{C}}

\begin{document}
\begin{center}
\Large\textbf{k-Local Folding: A Local Alignment Approach to RNA Folding}\\
\vspace*{0.8cm}
\large 
Ben Chugg \qquad
Coulter Beeson \qquad
Kenny Drabble \qquad
Jeffrey Jeyachandren 
\end{center}
\vspace*{0.3cm}

\section{Introduction}

Ribonucleic acids (RNA) play a crucial role in all living organisms, serving both as information storage as well as providing catalytic activity. Given their diverse functions, RNA come in many different varieties, such as mRNA encoding genetic information for translation into proteins, tRNA for the mapping of codons to amino acids, and ribozymes with catalytic activity such as ribosomes and spliceosomes (\cite{eight},\cite{nine},\cite{ten}), as well as numerous other less understood forms. As opposed to DNA, which is double stranded, RNA is often single stranded and forms --- usually complex ---  three dimensional structures by pairing with itself. As with proteins, the three dimensional structure of RNA is critical to its function, and structural prediction is a natural first step when aiming to ascribe function to a given RNA, as well as in the construction of synthetic sequences with novel properties ( \cite{one},\cite{two}). 

To predict the three dimensional structure of a given RNA sequence it is often necessary to first determine the secondary structure. RNA, as with proteins, will adopt structure(s) that minimize their total energy. The major stabilizing interaction for RNA comes from their intramolecular base pairing. That is, sequences of similar length base pair internally with other near palindromic sequences \cite{three}. Accordingly, most algorithmic approaches seek to maximize the number of these base pairings. Alternatively some approaches aim to measure other energetic interactions between bases, such as base stacking, and search for a structure of minimal global energy (\cite{four},\cite{five},\cite{eight}). Regardless of the approach used, most modern RNA folding algorithms use a similar recurrence that is amenable to dynamic programming. 

We present a modified heuristic approach to RNA folding which seeks to maximize the interactions among regions which are nearly palindromic. Our approach may be viewed as a preprocessing step to the typical RNA folding approach. We pair pair specific regions of the strand using a variant of local alignment, extract these regions from the original strand and run the usual dynamic programming algorithm on the remaining parts of the strand. 
\section{Preliminaries and Methodology}
\subsection*{Energy Minimization Model of RNA Folding}
RNA consists of the four base pairs Adenine (\A), Guanine (\G) , Cytosine (\C) and Uracil (\U). As opposed to DNA, the base pairs of RNA pair in a complementary fashion: Adenine to Uracil ($\A-\U$) and Cytosine to Guanine ($\C-\G$). 

There are several approaches to trying to determine the secondary structure (which bases are matched with which) of RNA.  A common approach is energy minimization (\cite{three}) which is formulated as a dynamic program as follows. Let $s=s_1,\ldots,s_n$ be a strand of RNA, where $s_i\in\{\A,\C,\G,\U\}$ and let $S(i,j)$ denote the optimal score of folding the subsequence $s_i,s_{i+1}\ldots s_j\subset s$. Then, 
\[S(i,j)=\max\begin{cases}
S(i+1,j-1)+1,&\text{if }i,j\text{ are a base pair}\\
S(i+1,j)\\
S(i,j-1)\\
\max_{i<k<j}\{S(i,k)+S(k+1,j)\},&\text{bifurcation}
\end{cases}\]
\section{Results}

\section{Analysis}

\section{Extensions}
1. Run tests against folded RNA to see what value of $k$ best captures pseudo-knot structure. 
2. Probabilistic (viterbi) methods. 

\section{Final Remarks}

\bibliographystyle{plain}
\bibliography{ref.bib}


\end{document}
